\documentclass{article}
\usepackage[utf8]{inputenc}
\usepackage{amsmath}
\usepackage{graphicx}
\usepackage{hyperref}
\usepackage{titling}

\title{\textbf{\Huge LangChain Framework }}
\author{Name: Alireza Dastmalchi Saei}
\date{Stu No.: 993613026}

\pretitle{%
  \begin{center}
  \includegraphics[width = 150px]{Images/university-of-isfahan-logo.png}\\[\bigskipamount]
  \vspace{3cm}
}
\posttitle{\end{center}}

\begin{document}
\maketitle

\pagebreak

\tableofcontents

\pagebreak

\section{Introduction}

LangChain is a framework designed for developing applications powered by large language models (LLMs). It streamlines every phase of the LLM application lifecycle, including development, productionization, and deployment.\\\\
The aim of this presentation is to introduce class members to the LangChain framework and illustrate how to implement a chatbot using its features. Topics covered in this presentation include various types of memories in chatbots, Chains, Evaluation methods, and Agents.\\\\
In the second part, A step-by-step process of implementing a chatbot is provided, which includes Document Loading, Document Splitting, Vectorstores and Embeddings, Retrieval, QnA of custom document, and Chatting with Document.

\end{document}
