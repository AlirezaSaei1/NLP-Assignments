\documentclass{article}


\title{NLP Assignment 1 Answers}
\author{Alireza Dastmalchi Saei - 993613026}
\date{\today}

\begin{document}

\maketitle

\section{Questions}
\subsection{Q1 - Ambiguity}
\begin{enumerate}
    \item \textbf{Phonetic Ambiguity}: Phonetic ambiguity occurs when different words or phrases sound similar, leading to confusion in spoken language. For example, "night" and "knight" sound identical but have different meanings. Resolving phonetic ambiguity may require context or clarification from the speaker.

    \item \textbf{Lexical Ambiguity}: Lexical ambiguity arises when a single word has multiple meanings. For example, the word "bank" can refer to a financial institution or the side of a river. Resolving lexical ambiguity often involves considering the context in which the word is used.

    \item \textbf{Syntactic Ambiguity}: Syntactic ambiguity occurs when a sentence can be parsed in multiple ways due to the structure or arrangement of words. For example, in the sentence "I saw the man with the telescope," it's unclear whether "with the telescope" modifies "I saw" or "the man." Resolving syntactic ambiguity requires analyzing the grammatical structure of the sentence.

    \item \textbf{Pragmatic Ambiguity}: Pragmatic ambiguity arises from the context or implied meaning of a sentence, rather than its literal interpretation. It occurs when a statement could have multiple interpretations depending on the speaker's intentions or the context in which it's used. Resolving pragmatic ambiguity often involves understanding the speaker's communicative intent and the social or cultural context of the conversation.
\end{enumerate}


\end{document}
