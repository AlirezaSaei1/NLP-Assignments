\documentclass{article}
\usepackage[utf8]{inputenc}
\usepackage{amsmath}
\usepackage{graphicx}
\usepackage{titling}

\title{\textbf{\Huge NLP Assignment 1: Research}}
\author{Name: Alireza Dastmalchi Saei}
\date{Stu No.: 993613026}

\pretitle{%
  \begin{center}
  \includegraphics[width = 150px]{university-of-isfahan-logo.png}\\[\bigskipamount]
  \vspace{3cm}
}
\posttitle{\end{center}}

\begin{document}
\maketitle

\pagebreak

\section{Text Summarization}
Text Summarization is a natural language processing (NLP) task that involves condensing a lengthy text document into a shorter, more compact version while still retaining the most important information and meaning. The goal is to produce a summary that accurately represents the content of the original text in a concise form.\\\\
There are different approaches to text summarization:
\begin{itemize}
  \item \textbf{Extractive methods} that identify and extract important sentences or phrases from the text.
  \item \textbf{Abstractive methods} that generate new text based on the content of the original text.
\end{itemize}
\pagebreak

\section{Sentiment Analysis}
Sentiment analysis is a natural language processing (NLP) technique used to determine whether data is positive, negative or neutral. Sentiment analysis is often performed on textual data to help businesses monitor brand and product sentiment in customer feedback, and understand customer needs. Statistical machine learning models like following are used in sentiment analysis tasks:
\begin{itemize}
  \item Naive Bayes Classifier
  \item Support Vector Machine (SVM)
  \item Logistic Regression
  \item Random Forest
  \item Gradient Boosting Machines (GBM)
\end{itemize}

\pagebreak

\section{Machine Translation}
Machine translation (MT) in natural language processing (NLP) refers to the automated process of translating text or speech from one language to another. The primary goal of machine translation is to enable communication between people who speak different languages by automatically converting text or speech from a source language into an equivalent text or speech in a target language. There are 4 types of machine translation:
\begin{itemize}
  \item Statistical Machine Translation (SMT)
  \item Rule-based Machine Translation (RBMT)
  \item Hybrid Machine Translation (HMT)
  \item Neural Machine Translation (NMT)
\end{itemize}
\pagebreak

\section{Article: -----}
\end{document}
